\documentclass{article}

\usepackage{tikz}

\begin{document}

\begin{tikzpicture}[scale=1.5]
  \clip (-1,-1) rectangle (7,1);
  \draw[domain=0.2:5.8, samples=80] plot (\x,{ln(\x)/\x}) ;
  \draw (-1,0)--(5.8,0) (0,-2)--(0,1);
  \draw [densely dashed] (2.72,0) node[below]{$e$} -- (2.72,0.37);
  \draw (1,0) node[below right]{$1$} node{$|$};
  \draw (0,0) node[below right]{$0$} node{$|$};
  \draw (3,0) node[below right]{$3$} node{$|$};
  \draw (2,0) node[below right]{$2$} node{$|$};
\end{tikzpicture}

\begin{tikzpicture}[xscale=2,yscale=100]
  \draw[domain=1.85:4.5, samples=80] plot (\x,{ln(\x)/\x}) ; % la courbe
  \draw [dashed] (1.7,{ln(2)/2}) -- (4.5,{ln(2)/2}); % droite y = ln(2)/2
  \draw (1.7,{ln(2)/2})
    node[left]{$\displaystyle{y=\frac{\ln(2)}{2}}$}; % Equation de la droite
  \draw (2,{ln(2)/2}) % point d'abscisse 2
    node[below right] {$x=2$}
    node{$\bullet$};
  \draw (4,{ln(2)/2})  %  point d'abscisse 4
    node[below left] {$x=4$}
    node{$\bullet$}; 
  \draw (2.718,0.3678)  % point d'abscisse e
    node[above]{$\displaystyle{\left(e,\frac{1}{e}\right)}$}
    node{$\bullet$};
\end{tikzpicture}

\end{document}
